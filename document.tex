% 文档类
\documentclass[13pt]{ctexart}
% 设置页面
\usepackage{geometry}
% 插图片
\usepackage{graphicx}
% 设置标题 重命名为英文
\renewcommand{\figurename}{Figure}
\renewcommand{\tablename}{Table}
\renewcommand{\contentsname}{Contents}
% 设置摘要页缩减 
\usepackage{changepage}
% 便于修改字体
\usepackage{fontspec}
% 设置页眉页脚
\usepackage{fancyhdr}
% 清空页眉页脚
\pagestyle{fancy}
% 设置列表缩进
\usepackage[shortlabels]{enumitem}
% 设置修改默认的section标题大小
\usepackage{titlesec}
\titleformat*{\section}{\LARGE}
\titleformat*{\subsection}{\Large}
\titleformat*{\subsubsection}{\Large}
% 使用数学宏包
\usepackage{amsmath}
% 设置表格的列格式
\usepackage{array}
% 三线表宏包
\usepackage{booktabs}
% 设置产考文献不输出默认名
\usepackage{etoolbox}
\patchcmd{\thebibliography}{\section*{\refname}}{}{}{}
% 引入网站作为参考文献
\usepackage{url}
% 设置等宽的代码字体
\setmonofont{Courier New}  % windows 代码字体 可以用这个
% \setmonofont{}
% 颜色
\usepackage{xcolor}
% 代码高亮方案宏包
\usepackage{listings}
\definecolor{CPPLight}  {HTML} {686868}
\definecolor{CPPSteel}  {HTML} {888888}
\definecolor{CPPDark}   {HTML} {262626}
\definecolor{CPPBlue}   {HTML} {4172A3}
\definecolor{CPPGreen}  {HTML} {487818}
\definecolor{CPPBrown}  {HTML} {A07040}
\definecolor{CPPRed}    {HTML} {AD4D3A}
\definecolor{CPPViolet} {HTML} {7040A0}
\definecolor{CPPGray}  {HTML} {B8B8B8}
\lstset{
	basicstyle=\ttfamily,
	breaklines=true,
	framextopmargin=50pt,
	frame=bottomline,
	columns=fixed,       
    %numbers=left,                                       % 在左侧显示行号
	frame=none,                                          % 不显示背景边框
	backgroundcolor=\color[RGB]{255,255,255},            % 设定背景颜色
	keywordstyle=\color[RGB]{40,40,255},                 % 设定关键字颜色
	numberstyle=\footnotesize\color{darkgray},           % 设定行号格式
	commentstyle=\itshape\color[RGB]{0,96,96},                % 设置代码注释的格式
	stringstyle=\slshape\color[RGB]{128,0,0},   % 设置字符串格式
	showstringspaces=false,                              % 不显示字符串中的空格
	language=python,                                     % 设置语言
	morekeywords={alignas,continute,friend,register,true,alignof,decltype,goto,
		reinterpret_cast,try,asm,defult,if,return,typedef,auto,delete,inline,short,
		typeid,bool,do,int,signed,typename,break,double,long,sizeof,union,case,
		dynamic_cast,mutable,static,unsigned,catch,else,namespace,static_assert,using,
		char,enum,new,static_cast,virtual,char16_t,char32_t,explict,noexcept,struct,
		void,export,nullptr,switch,volatile,class,extern,operator,template,wchar_t,
		const,false,private,this,while,constexpr,float,protected,thread_local,
		const_cast,for,public,throw,std},
	emph={map,set,multimap,multiset,unordered_map,unordered_set,numpy,graph,path,append,extend,
		unordered_multiset,unordered_multimap,vector,string,list,deque,
		array,stack,forwared_list,iostream,memory,shared_ptr,unique_ptr,
		random,bitset,ostream,istream,cout,cin,endl,move,default_random_engine,
		uniform_int_distribution,iterator,algorithm,functional,bing,numeric,},
	emphstyle=\color{CPPViolet}, 
}

\begin{document}
\newgeometry{top = 1cm, right = 2.54cm, left = 2.54cm, bottom = 2.54cm}
% 第一页的字体为times new roman
\setmainfont{Times New Roman}
\thispagestyle{empty}

% 摘要页每年都变,所以单独拿出去了
\begin{table}[h]
    \quad { }  \begin{minipage}[t]{5.5cm}
        % arraystretch 是调节列高
        \begin{tabular}[t]{>{\centering\arraybackslash}b{10em}}
            \fontsize{12pt}{10pt}\selectfont \textbf{Problem Chosen}\\ [2pt]
            {\color{red} \fontsize{20pt}{10pt}\selectfont ABCDEF}
        \end{tabular}
    \end{minipage}
    \begin{minipage}[t]{5.2cm}
        \begin{tabular}[t]{>{\centering\arraybackslash}p{10em}}
            \fontsize{12pt}{10pt}\selectfont \textbf{2022} \\ [-2pt]
            \fontsize{12pt}{10pt}\selectfont \textbf{MCM/ICM} \\ [-2pt]
            \fontsize{12pt}{10pt}\selectfont \textbf{Summary Sheet}
        \end{tabular}
    \end{minipage}
    \begin{minipage}[t]{3cm}
        \begin{tabular}[t]{>{\centering\arraybackslash}b{12em}}
            \fontsize{12pt}{10pt}\selectfont \textbf{Team Control Number} \\ [2pt]
            {\color{red} \fontsize{21pt}{10pt}\selectfont 114514}
        \end{tabular}
    \end{minipage}
\end{table}
\vspace{-20pt}
\noindent{\rule{\textwidth}{0.5mm}}

% 标题
{\centering\fontsize{18}{16}\selectfont\textbf{{Analysis and Supression of Opioid Spread}}
% 摘要
\vspace{10pt} 

\fontsize{13}{10}\selectfont\textbf{{Summary}}\par}

\vspace{10pt}

% 正文字体 13 pt
\fontsize{13}{12.5}\selectfont

\begin{adjustwidth}{1cm}{1cm}
We propose a model to describe the characteristics of the number, rate and direction of opioid spread in and between states and counties using transition matrices. The result shows that among the five given states, Ohio is most likely the source of opioid cases and has frequent opioid transition with Pennsylvania and Kentucky. We also find that if no effective regulatory measures are imposed, the number of drug cases in Ohio will increase up to 141,266 in the next decade. The spread of opioids between counties was analyzed in the same way, and the result is shown in Figure 3.

We determine two epidemic thresholds based on characteristics of opioid spread, with respect to the number of opioid cases and opioid spread rate respectively. We perform simulations to forecast the number of opioid cases in states and counties. The result shows that the number of narcotic analgesics cases in Ohio reaches the threshold in 2022, and the number of heroin cases in Pennsylvania reached the threshold in 2023.

We analyze the correlations between the number of opioid cases and various socio-economic factors using information entropy. We find that the number of opioids cases demonstrates high correlations with disability status, educational level, family status and adolescents. Therefore, we provide suggestions based on these factors to help suppress opioid spread. We also evaluate the effectiveness of these strategies on both the spread rate and the number of opioid cases. The result shows that when the amount of opioids spread is reduced to 74.5\%, the amount of opioids will stay on a relatively low level, and become lower than that of 2010 after 5 years.

We also do sensitivity analysis of various parameters to prove the robustness of our model. The result shows that both the continuous and periodical inflows lead to increase in the number of opioids in the state, and the increased amount depends on the specific transition amount.

\vspace{15pt}
\textbf{keywords} : Transition matrix; Multi-level thresholds; Information entropy
\end{adjustwidth} 


% 开始写 memo 信
% 更换字体为 palatino 也可以不换
\setmainfont{texgyrepagella-regular.otf} % linux 字体设置
% \setmainfont{TeX Gyre Pagella} % windows 字体设置
\newpage
\newgeometry{left = 3.5cm, right = 3.5cm}
\thispagestyle{empty}

{\centering \fontsize{18pt}{14pt}\selectfont \textbf{MEMO}\par}

\noindent FROM: Team {} 1917694 , MCM C

\noindent To: The group of Governors

\noindent Date: January 28, 2019

\vspace{10pt}

Dear Officials:

It is our honor to help you with analyzing the spread and characteristics of opioid. We are writing this letter to report our findings.

We analyzed the characteristics of opioid spread in the five states and counties, and found that if left uncontrolled, the opioid abuse will continue to spread, and the number of drug users will increase. We also found that there is no obvious relationship between the spread of opioid use and the geographical location. We deduced that Ohio is most likely the originated state of opioid use, and the drug cases will increase significantly over time. Of all five states, West Virginia shows a relatively healthy status. We also analyzed each county in the five states. Taking West Virginia as an example, Monongalia and Berkeley may be the originating counties for heroin transmission, and drug cases in these two counties are mainly spread to Kanawha and Berkeley.

We determined two epidemic thresholds based on characteristics of opioid spread, with respect to the number of opioid cases and opioid spread rate respectively. We found that the number of narcotic analgesics cases in Ohio reaches the threshold in 2022, and the number of heroin cases in Pennsylvania reached the threshold in 2023.

We analyzed the correlations between the number of opioid cases and various socio-economic factors. We found that the number of opioids cases shows high correlations with disability status, educational level, family status and adolescents. Therefore, we provide suggestions based on these factors to help suppress opioid spread. We also evaluate the effectiveness of these strategies. The result shows that when the amount of drug spread is less than 26.3\% on before, the number of drug users will be significantly reduced in 3 years, and become lower than that of 2010 in 5 years. 

We also do sensitivity analysis of our model. We found that inflows of drugs from outside the states also plays an important role in affecting the number of opioid cases.

Based on our analysis, we provide some suggestions to effectively prevent opioid spread:

1. National and local governments play an important role in detecting and preventing opioid overdose and abuse. Monitoring systems and early warning mechanisms should be established to effectively detect and respond to the spread of opioid abuse.

2. Provide more job opportunities to alleviate people's pressure, even to set up a free subsidy program. The local government should also publicize the harm of opioids abuse.

3. Prescriptions and sales of opioids should be strictly controlled, and a feedback mechanism for mandatory postoperative hospital visits should be established. The patients should also be examined for abuse of opioids during the hospital examination.

4. Impose careful check and strict management of logistics, to reduce the impact of opioids from outside the states.

We believe that our model is useful for preventing opioid spread in the future. You are welcome to contact us at any time for further cooperation.

% 信多出一页,清理页眉页脚
\thispagestyle{empty}
% 信的结尾
{\raggedleft
Sincerely yours

MCM C Team 1917694\par
}

% 目录页
\newpage
\thispagestyle{empty}
\tableofcontents
\newpage

% 目录页后面是第一页
\setcounter{page}{1}

\input{chapters/model.tex}

\newpage
\section{\textbf*{References}\addcontentsline{toc}{section}{References}}
\fancyhf{}
\fancyhead[R]{ }
\fancyhead[L]{ }
\bibliography{books}
\Large
\bibliographystyle{IEEEtran}

\newpage
\section{\textbf*{Appendices}\addcontentsline{toc}{section}{Appendices for Code and Data}} 
\fontsize{10pt}{12.5pt}\selectfont
Here is Code we used in our model, which python is the main development language.
\vspace{7pt}
\subsection{\textbf*{Appendices A}}
\begin {figure}[h]
	\centering % 居中显示
	\includegraphics[width=15cm,height=12cm]{figure/8.png}
	\caption{Transition matrix for synthetic opioid spread rate in West Virginia} % 标题
\end {figure}

\subsection{\textbf*{Appendices B: Forecast and Draw the Drugs of 2018-2027 Cases}}
\noindent{\rule{\textwidth}{0.2mm}}
\vspace{-18pt} 
\fontsize{11pt}{12.5pt}\selectfont
{
	\lstinputlisting[language=python]{code/predict.py}
}
\vspace{-15pt}
\noindent{\rule{\textwidth}{0.2mm}}
\subsection{\textbf*{Appendices C: The Program for Calculating Information Gain}}
\noindent{\rule{\textwidth}{0.2mm}}
\vspace{-18pt} 
\fontsize{11pt}{12.5pt}\selectfont
{
	\lstinputlisting[language=python]{code/information.py}
}
\vspace{-15pt}
\noindent{\rule{\textwidth}{0.2mm}}
\subsection{\textbf*{Appendices D: Transition Matrix of Five States}}
\noindent{\rule{\textwidth}{0.2mm}}
\vspace{-18pt} 
\fontsize{11pt}{12.5pt}\selectfont
{
	\lstinputlisting[language=python]{code/states.py}
}
\vspace{-15pt}
\noindent{\rule{\textwidth}{0.2mm}}

\end{document}